\documentclass[12pt,a4paper]{report}
\usepackage{graphicx}
\usepackage{tabularx}
\usepackage{array}

\begin{document}
%--------------Title Page ------------------
\thispagestyle{empty}
\begin{center}
\textbf{\large{Merchant Monetary System}}\\
\vspace{0.5cm}
\textbf{ Data Structure} \\
\vspace{1.5cm}
\includegraphics[scale=.07]{UETLogo}\\
\vspace{1.5cm}
\underline{ Project Supervisor}\\
\vspace{0.5cm}
Mr. Samyan Qayyum Wahla\\
\vspace{1cm}
\underline {Group ID $(G 11)$} \\
\vspace{0.5cm}
Project Member\\
\vspace{0.5cm}
\begin{tabular}{ m{5cm} m{4cm}}
 Syed Hashir & 2021-CS-1 \\ 
 Kabir Ahmed & 2021-CS-4  \\  
 M. Hamad Hassan & 2021-CS-33
\end{tabular}
\vspace{2cm}
\par\rule{\textwidth}{0.5pt} 
Department of Computer Science\\
University of Engineering and Technology, Lahore\\
Pakistan
\end{center}
\newpage
%----------------Data Structure-----------
\chapter*{Data Strucuture}
The following section shows the reason for choosing the data structure in the particular use case with a brief explanation.
\section{Use Case 1:LogIn}
\begin{center}
\begin{tabular}{ | m{4cm}|m{12cm}| }\hline
\textbf{Use Case ID}& U01 \\ \hline
\textbf{Data Structure Used}& Linked List \\ \hline

\textbf{Time Complexity}& 
In Worst Case: Search: O(n), Insertion: O(1), Deletion: O(n)\\\hline
\textbf{Space Complexity}& O(n)\\\hline

\textbf{Pseudocode}& 
\textbf{Search:} \\&
LIST-SEARCH(L,k)\\&
1 x=L.head\\&
2 while x $\neq$ NIL and x:key $\neq$ k \\&
3\hspace{6 mm} x = x.next\\&
4 return x\\&
\textbf{Insert:} \\&
LIST-INSERT(L, x) \\&
1 x.next=L.head \\&
2 if L.head $\neq$ NIL \\&
3\hspace{6 mm} L.head.pre = x \\&
4 L.head = x \\&
5 x.pre = NIL \\&
\textbf{Delete:} \\&
LIST-DELETE(L,x)\\&
1 if x.pre $\neq$ NIL\\&
2\hspace{6 mm} x.pre.next=x.next\\&
3 else L.head D x.next\\&
4 if x.next $\neq$ NIL\\&
5\hspace{6 mm} x.next.pre =x.pre
 \\ \hline
 \end{tabular}
\begin{tabular}{ | m{4cm}|m{12cm}| }\hline
\textbf{Justification for the use of data structure}&
In mentioned use case required a linear-dynamic data structure. 
Doubly LinkedList provides an efficient way to search the specific information from a large amount of data and then compare it with input information to produce the required result. It allows you to move back and forth in the list to get the required result.
 \\ \hline
\textbf{Available choices}& Array List,Hash Table \\ \hline
\textbf{Comparison}&
The array list worst and average case time complexity is O(n). It takes contiguous memory. The hash table is best in the average case, but in the worst case time, complexity rise to O(n). It takes contiguous memory for storing the hash function value. In the average and worst case, the linked list insertion and deletion take O(1) time. In the average and worst case, it takes O(n) time for deletion. It did not require contiguous memory allocation. 
Array list, hash table, and linked list space complexity O(n) are the same.
Array list, hash table, and linked list space complexity O(n) are the same.


 \\ \hline



\end{tabular}
\end{center}
\newpage 
\begin{tabular}{ | m{2cm} | m{3cm}|m{9cm}| } \hline
\textbf{ Use Case ID}&	\textbf{Data Structure Used}	& \textbf{Justification for the use of data structure}  \\ \hline
U01 &Linked List& In the U01 (LOGIN), search and compare the user from the list so when the user data is found it returns the action.\\ \hline
U04 &Linked List& In the U04 (Account Details), Grid of the added users shown lists where all the users are stored(added) \\ \hline
U05 &Linked List& In the U05 (Update Account), Update data of  the user present in the Linked List\\ \hline
U06 &Linked List&In the U06 (Add Product), Add the product data in the List. \\ \hline
U07 &Linked List&In the U07 (View Product), View the product data in the Grid that are stored in the list at the backend. \\ \hline
U08 &Linked List& In the U08 (Update Product), Update the product data in the list where the data of the products are added.\\ \hline
U09 &Linked List& In the U09 (Add Rider), Add the Rider data in the List. Selection of the list is because there is the ease in the deletion and search in the data of list\\ \hline
U10 &Linked List&In the U10 (Update Rider), update the rider data. Selection of the list is to search is to ease. \\ \hline
U11 &Queue      & In the U11(Order Product), To place the order we use the mechanism of First in and First Out (first order item will be placed first)\\ \hline
U12 &Stack      & In the U12 (Email), To send the mail and view the mail (first send mail is shown in the last and the most recent one in the first)\\ \hline
U13 &Linked List& In the U13 (Add Warehouse), Add the detail data of the warehouse in the list.\\ \hline
\end{tabular}
\begin{tabular}{ | m{2cm} | m{3cm}|m{9cm}| } \hline
U14 &Linked List& In the U14 (Detail Warehouse), Select the desired warehouse and delete the data of the warehouse and also delete the data from the list and selection of list is that to delete the warehouse other indexes of list easily manage.
 
\\ \hline
U15 &Linked List&In the U15 (Edit Warehouse), Select the data from the list and Edit the detail data of the warehouse in the list. 
 
 \\ \hline
U16 &Linked List& In the U16 (Order Status), Data is selected and data of the desired Order is updated in the list.\\ \hline
U17 &BST        & In the U17 (Route Finder), Routes are found according to the points (nodes) so selection of BST is due to the ease of the data finding.\\ \hline
U18 &Linked List&In the U18 (Add Shopkeeper), Add the shopkeeper data in the list because there is an ease for the deletion and searching. \\ \hline
U19 &Linked List& In the U19 (Add Payment), payment of the specific shopkeeper is added on the list to search and edit the details in the list.\\ \hline
U20 &Linked List&In the U20 (Add Expenses Amount), Add the Expenses data in the List. Because there is an ease to update the specific data in the list and search or delete it in list. \\ \hline
U21 &Linked List&In the U21 (Create Account), Linked list is used to add user. \\ \hline

\end{tabular}
























\end{document}